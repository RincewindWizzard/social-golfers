Das Social Golfer Problem fragt nach einem Spielplan für einen Golfclub, der sich wöchentlich trifft, um in mehreren Gruppen zu golfen, wobei zwei Golfer nie mehr als einmal in der gleichen Gruppe spielen.
Dabei wird die Mitgliederzahl des Clubs, sowie die Gruppengröße und die Anzahl der Wochen variiert.
Die Berechnung einer (neuen) Lösung ist sehr komplex, da zu jeder Lösung eine große Menge an äquivalenten symmetrischen Lösungen existiert.
Das SGP hat sich zu einer Standardanwendung für Benchmarks im Bereich der Constraints Programmierung entwickelt.
Carlos Cotta et al stellen in ihrem Artikel \cite{cotta06} einen neuen Memetischen Algorithmus vor, um dieses Problem zu lösen. 
Sie beschränken sich dabei auf Selektion, Mutation und lokaler (Tabu-)Suche, da eine Rekombination mit hoher Warscheinlichkeit zu symmetrischen Lösungen führen würde.
Die Autoren führen an ihren Beispielalgorithmen verschiedene Performancetests durch und untersuchen den Einfluss einer Baldwinschen und einer Lamarckschen Lernstrategie.
