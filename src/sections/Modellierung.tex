Wie bereits beschrieben, ist die Aufgabenstellung $n = g \cdot s$ Golfer in $g$ Gruppen zu je $s$ Spielern in $w$ Wochen einzuteilen, so dass zwei Golfer nicht häufiger als einmal gemeinsam spielen. 
Eine Probleminstanz besteht also aus dem Tripel $(g, s, w)$. Eine (möglicherweise auch falsche) Lösung hat die Form:
\begin{gather*}
  \sigma : \N_g \times \N_w \rightarrow 2^{\N_n}
\end{gather*}
Mit der Einschränkung $|\sigma(g',w')| = s$, wobei $\N_i = \{1, 2, 3, ..., i\}$ für alle $i \in \N$ gilt. 
$\sigma$ ist also eine Funktion, die für ein Tupel mit der Woche und der Gruppe eine Menge von Spielern zurückgibt, die sich in dieser Gruppe befinden.
Notwendige Bedingungen für eine Lösung sind folgende:\\
Jeder Golfer spielt pro Woche genau einmal:
\begin{equation} 
  \forall p \in \N_n : \forall \omega \in \N_w : \exists !\che \in \N_g : p \in \sigma(\che, \omega)
\end{equation}
Anders ausgedrückt: Die Gruppen sind innerhalb einer Woche paarweise disjunkt:
\begin{equation}
  \begin{split}
    & \forall \omega \in \N_w : \forall \che, \che' \in \N_g:\\
    & \che \neq \che' \Rightarrow \sigma(\che, \omega)  	\cap \sigma(\che', \omega) = \emptyset
  \end{split}
\end{equation}
Im weiteren Verlauf werden wir die Notation $\gamma(p, \omega)$ benutzen, um die Nummer der Gruppe zu bezeichnen, in der der Spieler $p$ in der Woche $\omega$ spielt.\\
Je zwei Golfer spielen maximal einmal gemeinsam in der selben Gruppe:
\begin{equation}\begin{split}
  & \forall \omega, \omega' \in \N_w : \forall \che, \che' \in \N_g:\\
  & \che \neq \che' \Rightarrow |\sigma(\che, \omega) \cap \sigma(\che', \omega')| \leq 1 
  \label{maximale Sozialisation}
\end{split}\end{equation}
Wir definieren uns $\#_\sigma(p_1, p_2)$ als die Anzahl, wie oft Spieler $p_1$ und Spieler $p_2$ zusammen gespielt haben.
\begin{equation} 
  \#_\sigma(p_1, p_2) = \sum_{\che \in \N_g} \sum_{\omega \in \N_w}[\{p_1, p_2 \} \subseteq \sigma(\che, \omega)]
\end{equation}
Hierbei bezeichnet $[\;]$ die Iverson Klammern und es gilt $[true] = 1$ sowie $[false] = 0$. 
Um auszudrücken, wie weit wir von der Erfüllung dieser Bedingung entfernt sind, definieren wir eine Funktion $v_\sigma(p_1, p_2)$:

\begin{equation} 
  v_\sigma(p_1, p_2) = \max(0, \#_\sigma(p_1, p_2) - 1)
\end{equation}


\subsection{Symmetrievermeidung}
Für eine Lösung existieren sehr viele äquivalente Lösungen. Innerhalb einer Gruppe ist die Reihenfolge der Golfer beliebig.
Ebenso ist die Reihenfolge der Gruppen innerhalb einer Woche, sowie die Wochen an sich belanglos. Außerdem kann eine neue äquivalente Lösung erzeugt werden, indem man überall die Namen von zwei Golfern tauscht.
Wir stellen also fest, dass zu jeder Lösung $(s!)^{g \cdot w} (g!)^w w!(gs)!$ symmetrische Lösungen existieren.
Insgesamt existieren für eine Instanz $(g,s,w)$ bei Darstellung in Form einer Liste, wie in \ref{Lösungsstring}, $(g \cdot s)!^w$ mögliche, teilweise symmetrische, Lösungen. 
Vergleiche auch mit $(g \cdot s \cdot w)!$, was die Anzahl aller Permutationen der selben Liste ist.

Die schwierigste bis jetzt gelöste Instanz $(8,4,10)$ \cite{agu04} hat insgesamt $159 \cdot 10^{352}$ mögliche Permutationen, 
welche jeweils in Äquivalenzklassen der Größe $283 \cdot 10^{196}$ unterteilt sind. Es gibt also $561 \cdot 10^{153}$ unterschiedliche Äquivalenzklassen von möglichen Lösungen.


Um zwei Lösungen unterscheiden zu können, braucht es unterschiedliche Strategien, um diese Symmetrien zu vermeiden. 
Die Symmetrie innerhalb einer Gruppe wird vermieden, indem man Mengen anstatt Listen verwendet. Diese Mengen werden innerhalb einer Woche anhand ihres kleinsten Elementes geordnet.
\begin{equation}
  \begin{split}
    & \sigma(\che, \omega) \prec \sigma(\che + 1, \omega) \\
    & \Leftrightarrow \min(\sigma(\che, \omega)) < \min(\sigma(\che + 1, \omega))
  \label{Gruppenordnung}
  \end{split}
\end{equation}
Wochen werden anhand des zweitkleinsten Elementes der ersten Gruppe geordnet.
\begin{equation}
  \begin{split}
    & \omega_i \prec \omega_{i+1} \\
    & \Leftrightarrow \min(\sigma(1, i) \setminus \{ 1 \}) < \min(\sigma(1, i + 1) \setminus \{ 1 \})
  \label{Wochenordnung}
  \end{split}
\end{equation}
Die Symmetrien, welche durch Umbenennung enstehen, wurden in dem Artikel nicht weiter behandelt. In \cite{sellmann02} wird dieses Thema weiter behandelt.


