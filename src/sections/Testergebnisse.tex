Um die These der Autoren zu prüfen, habe ich eine eigene Implentierung des beschriebenen Algorithmus vorgenommen, wobei ich bemüht war, möglichst genau das beschriebene Verfahren nach zu bilden.
Genau wie die Autoren nutze ich eine Größe von 25 Individuen für den Genpool, aus denen gewählt wird. Die lokale Suche wird mit $maxIter = g \cdot w$ beschränkt, 
damit die Chance besteht, dass die resultierende Lösung sich in jeder Gruppe von der Originalen unterscheidet. 
Ich lasse automatisiert Lösungen für verschiedene Probleminstanzen erzeugen und werde alle Ergebnisse, die bis zu meinem Vortrag erzeugt wurden, veröffentlichen. 
Ich kann verifizieren, dass die Autoren ein Verfahren vorgestellt haben, welches tatsächlich das versprochene Verhalten zeigt.
Über das Zeitverhalten kann ich kaum qualifizierte Aussagen machen, da dies sehr subjektiv und abhängig von der verwendeten Hardware ist. 
Die anspruchsvollste Lösung für eine Instanz, die ich finden konnte, ist die von Alejandro Aguado\cite{agu04} für $(8, 4, 10)$.
Sofern mein Programm eine ebenbürtige Lösung findet, werde ich diese auf Äquivalenz prüfen.
