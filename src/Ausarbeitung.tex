\documentclass[draft]{tcs-seminar}

\usepackage[utf8]{inputenc}  % Erlaubt es, Umlaute etc. zu verwenden (Datei muss UTF-8 Kodierung haben)
\usepackage[ngermanb]{babel} % Deutsche Übersetzung und Silbentrennung (Neue Rechtschreibung)
\usepackage{url}

\begin{document}

\title{Die Memetik des Einlochens - Eine Tabu-Suche}
\author{Dennis Lindemann}

\begin{abstract}
  Die Zusammenfassung beschreibt möglichst knapp den wesentlichen Inhalt
  und die wichtigsten in der Ausarbeitung vorgestellten Ergebnisse.
  Detaillierte Problem- oder Methodenbeschreibungen sowie Literaturverweise
  sind nicht Teil der Zusammenfassung.
  
  Hier dürfen Fachbegriffe aus dem jeweiligen Themenbereich,
  sofern notwendig, ohne Erläuterung verwendet werden.
  Vollständig neue Konzepte sollten jedoch erläutert werden.
  Einem Leser, der den Themenbereich der Arbeit kennt,
  soll so ermöglicht werden, sehr schnell den Inhalt der Arbeit zu erfassen.

\end{abstract}

\maketitle


\section{Einleitung}

Die Einleitung soll den Leser an das Thema der Ausarbeitung heranführen.
Sie sollte im Wesentlichen drei Fragen beantworten:

\begin{quote}
  Worum geht es?
\end{quote}
Zunächst wird, normalerweise in relativ leicht verständlichen Worten,
die behandelte Problemstellung erläutert.
Gut ist es, die Problemstellung an dieser Stelle auch zu motivieren.

\begin{quote}
  Was war schon bekannt?
\end{quote}
Danach kann die Arbeit in das übergeordnete Thema eingeordnet werden.
Sofern vorhanden, sollten dabei
die wichtigsten Vorarbeiten knapp dargestellt werden.
Dadurch soll dem Leser der Kontext der Arbeit vermittelt werden.

\begin{quote}
  Was ist neu?
\end{quote}
Schließlich muss der Inhalt der Arbeit,
insbesondere die erzielten Ergebnisse, beschrieben werden.
Hierbei sollte auch erwähnt werden,
wie sich die verwendeten Modelle, Methoden und Ergebnisse
von den zuvor genannten Vorarbeiten unterscheiden.

Diese Informationen müssen nicht zwingend
in genau dieser Reihenfolge und getrennt voneinander dargeboten werden.
Häufig ist es zum Verständnis der Problemstellung
oder der Ergebnisse notwendig, auch einige Begriffe einzuführen.
Dies kann an jeder Stelle passieren,
sollte sich aber in der Einleitung auf das Notwendigste beschränken
--~gegebenenfalls sind hier Beschreibungen in natürlicher Sprache
den formalen vorzuziehen.
Gerade die Einleitung sollte ansprechend geschrieben sein
und ,,Lust auf mehr'' machen.


\section{Der Inhalt}

Nach der Einleitung folgen ein oder mehrere Abschnitte,
die den Hauptteil der Ausarbeitung beinhalten.

Sofern dies in der Einleitung nicht möglich war,
wird im ersten Abschnitt die Problemstellung und das verwendete Modell
vollständig und formal korrekt vorgestellt.
In seltenen Fällen ist es zunächst nötig,
eine Art Einführung in den Themenbereich zu geben,
bevor eine umfassende Beschreibung möglich ist.

Danach folgt eine Beschreibung der Ergebnisse und wie diese erreicht wurden.


\section{Die Struktur}
\label{sec:struktur}

Stellt die Arbeit mehrere Ergebnisse dar,
können diese geeignet auf Abschnitte oder Unterabschnitte verteilt werden.

\subsection{Orientierung}

Der Inhalt sollte ein gute und vom Leser klar erkennbare Struktur haben.
Eine Aufteilung in Abschnitte und Unterabschnitte erlaubt es dem Leser,
sich besser in der Arbeit zu orientieren.

\subsection{Leserführung}

Dazu gehört auch, dass der Inhalt jedes (unter-) Abschnitts
entweder zu Beginn als eine Art Einleitung
oder durch den vorhergehenden Text ausreichend beschrieben und motiviert wird.

\subsubsection{Ziel}

Der Leser sollte zu jedem Zeitpunkt wissen,
was er gerade liest, und inwiefern das für die gesamte Arbeit relevant ist.

\subsubsection{Grenzen}

Dazu können bis zu drei Gliederungsebenen verwendet werden.
In den meisten Fällen sind allerdings zwei
(bis \textsf{\textbackslash subsection}) völlig ausreichend.


\section{Die Darstellung}

\begin{figure*}[t]
  \centering
  \rule{12cm}{2cm}
  \caption{Ein breiter, schwarzer Kasten}
  \label{abb:breiter-kasten}
\end{figure*}
Grundsätzlich sollte Klarheit und Verständlichkeit bei der Erstellung eines
informativen Textes die höchste Priorität haben.
Neben dem Aufbau des Textes sollte daher
z.B. auf die Lesbarkeit geachtet werden.
Sind die Abstände ausreichend groß,
aber nicht so groß, dass sie irritierend wirken?
Wurden ausreichend Absätze verwendet, die den Text auch optisch untergliedern
und dem Auge die Orientierung erleichtern (siehe auch Abschnitt~\ref{sec:struktur})?
Werden Änderungen von Schriftgröße und -schnitt spärlich verwendet,
und transportieren sie jeweils eine semantische Bedeutung?

Wichtig ist hierbei vor allem \emph{Konsistenz}.
Gleiche Begriffe oder Konzepte sollten innerhalb des Textes
immer auf dieselbe Art und Weise dargestellt werden.
Insbesondere bei Fachbegriffen muss auf eine korrekte
und konsistente Verwendung geachtet werden.
Als anderes Beispiel wurden in diesem Text alle Befehle
in serifenloser Schrift (mittels \textsf{\textbackslash textsf} gesetzt.
Dies wurde mit \emph{allen} solchen Befehlen gemacht,
und diese Schrift wurde nicht mit weiteren Bedeutungen überladen.

Auf der anderen Seite sollte der Text auch konsistent mit den Erwartungen
des Lesers sein, sich also an kulturellen Normen und Erwartungen halten.
So ist es zum Beispiel üblich, dass die erste Zeile eines Absatzes
einen größeren Zeileneinzug hat.
Diese Darstellung sollte daher ebenfalls verwendet werden,
und ist auch durch die Dokumentenklasse voreingestellt.
Dagegen sollten Zeilenumbrüche mittels \textsf{\textbackslash \textbackslash}
oder \textsf{\textbackslash newline} nur selten verwendet werden,
und dann nur, wenn eine andere Bedeutung, als die eines Absatzes,
transportiert werden soll.

Ein weiterer Aspekt der Darstellung sind Bilder
und andere visuelle Unterstützungen.
Diese können und sollen verwendet werden,
wenn Sie das Verständnis des Textes erleichtern.
Ein schwarzer Kasten wie in Abbildung~\ref{abb:kasten} ist dagegen nicht
besonders aufschlussreich, und sollte daher weggelassen werden.
\begin{figure}
  \centering
  \rule{3cm}{2cm}
  \caption{Ein schwarzer Kasten}
  \label{abb:kasten}
\end{figure}
Falls nötig, können sogar große Bilder wie Abbildung~\ref{abb:breiter-kasten}
verwenden.


\section{Der Abschluss}

Das Schlusswort der Ausarbeitung kann vielfältig gestaltet werden.
Am Ende der Arbeit sollten die erzielten Ergebnisse noch einmal zusammengefasst
und, durchaus auch kritisch, reflektiert werden.
Offen gebliebene oder neu entstandene Fragen
können hier ebenfalls formuliert werden.

Ganz am Ende der Arbeit sollte eine Liste
der verwendeten Literatur angegeben werden.
Es sind nur Quellen anzugeben,
aus denen Begriffe oder Erkenntnisse verwendet wurden
--~dieses ist dann an der entsprechenden Stelle mittels des Befehls
\textsf{\textbackslash cite}, siehe z.B. \cite{oetiker06}, zu kennzeichnen.
Zum Verständnis oder wegen einer besseren Erläuterung verwendeten Arbeiten
brauchen nicht zitiert werden.
\paragraph{Wichtig.}
Damit Referenzen (und auch Verweise auf Bilder, Abschnitte etc.)
korrekt angezeigt werden, muss das Dokument ein zweites Mal kompiliert werden.

\begin{thebibliography}{9}  
  \bibitem{cotta06}
    Carlos Cotta, Ivan Dotu, Antonio J. Fernandez, Pascal Van Hentenryck,
    \emph{Scheduling Social Golfers with Memetic Evolutionary Programming}.
    Springer-Verlag Berlin Heidelberg,
    2006.

  \bibitem{fahle01}
    Fahle, T. Schamberger, S., Sellman, M.
    \emph{Symmetry breaking}.
    In Walsh T., ed.: 7th International Conference on Principles and Practice of Constraint Programming. Volume 2239 of Lecture Notes in Computer Science., Paphos, Cyprus,
    Springer,
    2001 S93-107

  \bibitem{smith01}
    Smith, B.M.
    \emph{Reducing Symmetry in a combinatorial design problem}.
    In: Third International Workshop on the Integration of AI and OR Techniques in Constraint Programming for Combinatorial Optimization Problems,
    2001 S351-359

  \bibitem{sellmann02}
    Sellmann, M., Harvey, W.
    \emph{Heuristic constraint propagation}.,
    In Hetenryck, P. V., ed.: 8th International Conference on Principles and Practice of Constraint Programming. Volume 2470 of Lecture Notes in Computer Science,
    Ithaca, NY, USA
    Springer,
    2002 S738-743

  \bibitem{triska08}
    Triska, Markus and Musliu, Nysret
    \emph{An improved SAT formulation for the social golfer problem},
    In: Annals of Operations Research,
    Springer Verlag,
    2012


    
\end{thebibliography}

\end{document}
