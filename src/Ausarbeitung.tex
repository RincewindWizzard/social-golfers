\documentclass[draft]{tcs-seminar}

\usepackage[utf8]{inputenc}  % Erlaubt es, Umlaute etc. zu verwenden (Datei muss UTF-8 Kodierung haben)
\usepackage[ngermanb]{babel} % Deutsche Übersetzung und Silbentrennung (Neue Rechtschreibung)
\usepackage{url}
\usepackage[autostyle=true,german=quotes]{csquotes}

\begin{document}

\title{Ein Memetischer Algorithmus für das Social Golfer Problem}
\author{Dennis Lindemann}

\begin{abstract}
  Die Zusammenfassung beschreibt möglichst knapp den wesentlichen Inhalt
  und die wichtigsten in der Ausarbeitung vorgestellten Ergebnisse.
  Detaillierte Problem- oder Methodenbeschreibungen sowie Literaturverweise
  sind nicht Teil der Zusammenfassung.
  
  Hier dürfen Fachbegriffe aus dem jeweiligen Themenbereich,
  sofern notwendig, ohne Erläuterung verwendet werden.
  Vollständig neue Konzepte sollten jedoch erläutert werden.
  Einem Leser, der den Themenbereich der Arbeit kennt,
  soll so ermöglicht werden, sehr schnell den Inhalt der Arbeit zu erfassen.

\end{abstract}

\maketitle


\section{Einleitung}
  Das Social Golfer Problem (SGP) findet seine erste Erwähnung in leicht abgewandelter Form schon 1850 im Lady's and Gentleman's Diary, wo Reverend Thomas Kirkman folgende Frage stellte:

\begin{quote}
Fünfzehn junge Mädchen gehen in der Schule sieben Tage lang in Dreiergruppen nebeneinander. Stellen Sie die Gruppen so zusammen, dass zwei Mädchen nicht zweimal zusammen gehen.
\end{quote}
Diese Frage ist etwas spezieller als diejenige, welche in diesen Artikel untersucht wird:
\begin{quote}
$n$ Golfern spielen in $g$ Gruppen der Größe $s$ $w$ Wochen lang. Erstellen Sie einen Spielplan, in dem zwei Spieler maximal einmal zusammen spielen.
\end{quote}
Die allgemeine Variante hat internationale Bekanntheit erlangt, nachdem sie im Mai 1998 in der Newsgroup \texttt{sci.op-research} veröffentlicht wurde.

Das allgemeine SGP ist für einige Probleminstanzen hinreichend komplex, dass in absehbarer Zeit keine Lösung erwartet wird. Es liegt die Vermutung nahe, dass es sich hierbei um ein NP-vollständiges\footnote{->Entweder erklären oder Satz streichen.} Problem handelt.
Erschwerend kommt hinzu, dass Lösungen für das SGP eine hohe Symmetriedichte aufweisen. Beispielsweise können innerhalb eines Spielplanes die Wochen getauscht werden, sowie die Gruppen innerhalb einer Woche. Auch können zwei Spieler vertauscht werden, ohne dass sich die Lösung prinzipiell von der Alten unterscheidet. Jeder Lösungsansatz, der sich nicht eingehend mit dieser Erschwernis befasst, ist dazu verdammt, immer wieder redundante Lösungen zu finden.

Vor diesem Hintergrund ist ein Algorithmus, der für eine Probleminstanz $(g,s,w)$ in vertretbarer Zeit einen Spielplan erzeugt, von großem Interesse.
Mehrere Ansätze, zum Beispiel aus der Constraints Programmierung (z.B. \cite{fahle01}, \cite{smith01}, \cite{sellmann02}) oder mit Hilfe von SAT Solvern (z.B. \cite{triska08}), wurden schon verfolgt. Allerdings war unter den bisherigen Algorithmen noch keiner, mit einem evolutionären Ansatz.
Die Kombination von Selektion, Mutation und lokaler Suche, welche als \enquote{memetischer Algorithmus} bezeichnet wird, führt zu einer Performance, die sich mit den Alternativen messen kann.





\section{Mathematische Modellierung}
    Wie bereits beschrieben, ist die Aufgabenstellung $n = g \cdot s$ Golfer in $g$ Gruppen zu je $s$ Spielern in $w$ Wochen einzuteilen, so dass zwei Golfer nicht häufiger als ein mal gemeinsam spielen. Eine Probleminstanz besteht also aus dem Tripel $(g, s, w)$ ein (möglicherweise auch falsche) Lösung hat die Form:
\begin{gather*}
  \sigma : \N_g \times \N_w \rightarrow 2^{\N_n}
\end{gather*}
Mit der Einschränkung $|\sigma(g',w')| = s$. Wobei $\N_i = \{1, 2, 3, ..., i\}$ für alle $i \in \N$ gilt. $\sigma$ is also eine Funktion, die für ein Tupel mit der Woche und der Gruppe eine Menge von Spielern zurückgibt, die sich in dieser Gruppe befinden.
Notwendige Bedingungen für eine Lösung sind folgende:\\
Jeder Golfer spielt pro Woche genau einmal:
\begin{equation} 
  \forall p \in \N_n : \forall \omega \in \N_w : \exists !\che \in \N_g : p \in \sigma(\che, \omega)
\end{equation}
Anders ausgedrückt: Die Gruppen sind innerhalb einer Woche paarweise disjunkt:
\begin{equation}
  \begin{split}
    & \forall \omega \in \N_w : \forall \che, \che' \in \N_g:\\
    & \che \neq \che' \Rightarrow \sigma(\che, \omega)  	\cap \sigma(\che', \omega) = \emptyset
  \end{split}
\end{equation}
Im weiteren Verlauf werden wir die Notation $\gamma(p, \omega)$ benutzen, um die Nummer der Gruppe zu bezeichnen, in der der Spieler $p$ in der Woche $\omega$ spielt.\\
Je zwei Golfer spielen maximal einmal gemeinsam in der selben Gruppe:
\begin{equation}\begin{split}
  & \forall \omega, \omega' \in \N_w : \forall \che, \che' \in \N_g:\\
  & \che \neq \che' \Rightarrow |\sigma(\che, \omega) \cap \sigma(\che', \omega')| \leq 1
\end{split}\end{equation}
Wir definieren uns $\#_\sigma(p_1, p_2)$ als die Anzahl, wie oft Spieler $p_1$ und Spieler $p_2$ zusammen gespielt haben.
\begin{equation} 
  \#_\sigma(p_1, p_2) = \sum_{\che \in \N_g} \sum_{\omega \in \N_w}[\{p_1, p_2 \} \subseteq \sigma(\che, \omega)]
\end{equation}
Hierbei bezeichnet $[\;]$ die Iverson Klammern und es gilt $[true] = 1$ sowie $[false] = 0$. 
Um auszudrücken, wie weit wir von der Erfüllung dieser Bedingung entfernt sind, definieren wir eine Funktion $v_\sigma(p_1, p_2) = \max(0, \#_\sigma(p_1, p_2) - 1)$.

\subsection{Symmetrievermeidung}
Für eine Lösung existieren sehr viele äquivalente Lösungen. Innerhalb einer Gruppe ist die Reihenfolge der Golfer beliebig.
Ebenso ist die Reihenfolge der Gruppen innerhalb einer Woche, sowie die Wochen an sich belanglos. Außerdem kann einer neue äquivalente Lösung erzeugt werden, indem man überall die Namen von zwei Golfern tauscht. Wir stellen also fest, dass jede Lösung $(s!)^{g \cdot w} (g!)^w w!(gs)$ symmetrische Lösungen besitzt. 

Um zwei Lösungen unterscheiden zu können, braucht es unterschiedliche Strategien, um diese Symmetrien zu brechen. 
Die Symmetrie innerhalb einer Gruppe wird vermieden, indem man Mengen anstatt Listen verwendet. Diese Mengen werden innerhalb einer Woche anhand ihres kleinsten Elementes geordnet.
\begin{equation}
  \begin{split}
    & \sigma(\che, \omega) \prec \sigma(\che + 1, \omega) \\
    & \Leftrightarrow \min(\sigma(\che, \omega)) < \min(\sigma(\che + 1, \omega))
  \end{split}
\end{equation}
Wochen werden anhand des zweitkleinsten Elementes der ersten Gruppe geordnet.
\begin{equation}
  \begin{split}
    & \omega_i \prec \omega_{i+1} \\
    & \Leftrightarrow \min(\sigma(1, i) \setminus \{ 1 \}) < \min(\sigma(1, i + 1) \setminus \{ 1 \})
  \end{split}
\end{equation}
Die Symmetrien, welche durch Umbenennung enstehen, wurden in dem Artikel nicht weiter behandelt. In \cite{sellmann02} wird dieses Thema weiter behandelt.



\section{Evolutionärer Algorithmus}
  \subsection{Mutationen}
    Die Mutationen kann man als einen weiten Sprung zu einer anderen Lösung ansehen. (Makrooptimierung)
  \subsection{Tabu Suche}
    Die Tabu Suche ist eher ein kleiner Sprung, um eine bestehende (ungültige) Lösung zu verfeinern. (Mikrooptimierung)

\section{Testergebnisse}
Hier werden die Ergebnisse meiner Performancetests vorgestellt und mit den Ergebnissen aus dem Paper verglichen. Ein besonderes Augenmerk werde ich hierbei auf Laufzeit und Ressourcenverbrauch im Sinne der Landau Notation legen.



\section{Fazit}
Hier steht, was ich von der ganzen Sache halte und welche Ergebnisse ich aus dieser Arbeit gezogen habe.

\begin{thebibliography}{9}  
  \bibitem{cotta06}
    Carlos Cotta, Ivan Dotu, Antonio J. Fernandez, Pascal Van Hentenryck,
    \emph{Scheduling Social Golfers with Memetic Evolutionary Programming}.
    Springer-Verlag Berlin Heidelberg,
    2006.

  \bibitem{fahle01}
    Fahle, T. Schamberger, S., Sellman, M.
    \emph{Symmetry breaking}.
    In Walsh T., ed.: 7th International Conference on Principles and Practice of Constraint Programming. Volume 2239 of Lecture Notes in Computer Science., Paphos, Cyprus,
    Springer,
    2001 S93-107

  \bibitem{smith01}
    Smith, B.M.
    \emph{Reducing Symmetry in a combinatorial design problem}.
    In: Third International Workshop on the Integration of AI and OR Techniques in Constraint Programming for Combinatorial Optimization Problems,
    2001 S351-359

  \bibitem{sellmann02}
    Sellmann, M., Harvey, W.
    \emph{Heuristic constraint propagation}.,
    In Hetenryck, P. V., ed.: 8th International Conference on Principles and Practice of Constraint Programming. Volume 2470 of Lecture Notes in Computer Science,
    Ithaca, NY, USA
    Springer,
    2002 S738-743

  \bibitem{triska08}
    Triska, Markus and Musliu, Nysret
    \emph{An improved SAT formulation for the social golfer problem},
    In: Annals of Operations Research,
    Springer Verlag,
    2012


    
\end{thebibliography}

\end{document}
