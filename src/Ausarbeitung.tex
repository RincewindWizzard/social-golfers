\documentclass[draft]{tcs-seminar}

\usepackage[utf8]{inputenc}  % Erlaubt es, Umlaute etc. zu verwenden (Datei muss UTF-8 Kodierung haben)
\usepackage[OT2, T1]{fontenc}
\usepackage[russian,ngerman]{babel}
\usepackage{url}
\usepackage{amsmath}
\usepackage[autostyle=true,german=quotes]{csquotes}
\usepackage{color}

%\usepackage{breqn}
% convienience
\newcommand{\N}{\mathbb{N}}
\newcommand{\che}{\text{\foreignlanguage{russian}{ч}}}
\newcommand\todo[1]{\textcolor{red}{#1}}

\begin{document}

\title{Ein Memetischer Algorithmus für das Social Golfer Problem}
\author{Dennis Lindemann}

\begin{abstract}
  Die Zusammenfassung beschreibt möglichst knapp den wesentlichen Inhalt
  und die wichtigsten in der Ausarbeitung vorgestellten Ergebnisse.
  Detaillierte Problem- oder Methodenbeschreibungen sowie Literaturverweise
  sind nicht Teil der Zusammenfassung.
  
  Hier dürfen Fachbegriffe aus dem jeweiligen Themenbereich,
  sofern notwendig, ohne Erläuterung verwendet werden.
  Vollständig neue Konzepte sollten jedoch erläutert werden.
  Einem Leser, der den Themenbereich der Arbeit kennt,
  soll so ermöglicht werden, sehr schnell den Inhalt der Arbeit zu erfassen.

\end{abstract}

\maketitle


\section{Einleitung}
  Das Social Golfer Problem (SGP) findet seine erste Erwähnung in leicht abgewandelter Form schon 1850 im Lady's and Gentleman's Diary, wo Reverend Thomas Kirkman folgende Frage stellte:

\begin{quote}
Fünfzehn junge Mädchen gehen in der Schule sieben Tage lang in Dreiergruppen nebeneinander. Stellen Sie die Gruppen so zusammen, dass zwei Mädchen nicht zweimal zusammen gehen.
\end{quote}
Diese Frage ist etwas spezieller als diejenige, welche in diesen Artikel untersucht wird:
\begin{quote}
$n$ Golfern spielen in $g$ Gruppen der Größe $s$ $w$ Wochen lang. Erstellen Sie einen Spielplan, in dem zwei Spieler maximal einmal zusammen spielen.
\end{quote}
Die allgemeine Variante hat internationale Bekanntheit erlangt, nachdem sie im Mai 1998 in der Newsgroup \texttt{sci.op-research} veröffentlicht wurde.

Das allgemeine SGP ist für einige Probleminstanzen hinreichend komplex, dass in absehbarer Zeit keine Lösung erwartet wird. Es liegt die Vermutung nahe, dass es sich hierbei um ein NP-vollständiges\footnote{->Entweder erklären oder Satz streichen.} Problem handelt.
Erschwerend kommt hinzu, dass Lösungen für das SGP eine hohe Symmetriedichte aufweisen. Beispielsweise können innerhalb eines Spielplanes die Wochen getauscht werden, sowie die Gruppen innerhalb einer Woche. Auch können zwei Spieler vertauscht werden, ohne dass sich die Lösung prinzipiell von der Alten unterscheidet. Jeder Lösungsansatz, der sich nicht eingehend mit dieser Erschwernis befasst, ist dazu verdammt, immer wieder redundante Lösungen zu finden.

Vor diesem Hintergrund ist ein Algorithmus, der für eine Probleminstanz $(g,s,w)$ in vertretbarer Zeit einen Spielplan erzeugt, von großem Interesse.
Mehrere Ansätze, zum Beispiel aus der Constraints Programmierung (z.B. \cite{fahle01}, \cite{smith01}, \cite{sellmann02}) oder mit Hilfe von SAT Solvern (z.B. \cite{triska08}), wurden schon verfolgt. Allerdings war unter den bisherigen Algorithmen noch keiner, mit einem evolutionären Ansatz.
Die Kombination von Selektion, Mutation und lokaler Suche, welche als \enquote{memetischer Algorithmus} bezeichnet wird, führt zu einer Performance, die sich mit den Alternativen messen kann.





\section{Mathematische Modellierung}
    Wie bereits beschrieben, ist die Aufgabenstellung $n = g \cdot s$ Golfer in $g$ Gruppen zu je $s$ Spielern in $w$ Wochen einzuteilen, so dass zwei Golfer nicht häufiger als ein mal gemeinsam spielen. Eine Probleminstanz besteht also aus dem Tripel $(g, s, w)$ ein (möglicherweise auch falsche) Lösung hat die Form:
\begin{gather*}
  \sigma : \N_g \times \N_w \rightarrow 2^{\N_n}
\end{gather*}
Mit der Einschränkung $|\sigma(g',w')| = s$. Wobei $\N_i = \{1, 2, 3, ..., i\}$ für alle $i \in \N$ gilt. $\sigma$ is also eine Funktion, die für ein Tupel mit der Woche und der Gruppe eine Menge von Spielern zurückgibt, die sich in dieser Gruppe befinden.
Notwendige Bedingungen für eine Lösung sind folgende:\\
Jeder Golfer spielt pro Woche genau einmal:
\begin{equation} 
  \forall p \in \N_n : \forall \omega \in \N_w : \exists !\che \in \N_g : p \in \sigma(\che, \omega)
\end{equation}
Anders ausgedrückt: Die Gruppen sind innerhalb einer Woche paarweise disjunkt:
\begin{equation}
  \begin{split}
    & \forall \omega \in \N_w : \forall \che, \che' \in \N_g:\\
    & \che \neq \che' \Rightarrow \sigma(\che, \omega)  	\cap \sigma(\che', \omega) = \emptyset
  \end{split}
\end{equation}
Im weiteren Verlauf werden wir die Notation $\gamma(p, \omega)$ benutzen, um die Nummer der Gruppe zu bezeichnen, in der der Spieler $p$ in der Woche $\omega$ spielt.\\
Je zwei Golfer spielen maximal einmal gemeinsam in der selben Gruppe:
\begin{equation}\begin{split}
  & \forall \omega, \omega' \in \N_w : \forall \che, \che' \in \N_g:\\
  & \che \neq \che' \Rightarrow |\sigma(\che, \omega) \cap \sigma(\che', \omega')| \leq 1
\end{split}\end{equation}
Wir definieren uns $\#_\sigma(p_1, p_2)$ als die Anzahl, wie oft Spieler $p_1$ und Spieler $p_2$ zusammen gespielt haben.
\begin{equation} 
  \#_\sigma(p_1, p_2) = \sum_{\che \in \N_g} \sum_{\omega \in \N_w}[\{p_1, p_2 \} \subseteq \sigma(\che, \omega)]
\end{equation}
Hierbei bezeichnet $[\;]$ die Iverson Klammern und es gilt $[true] = 1$ sowie $[false] = 0$. 
Um auszudrücken, wie weit wir von der Erfüllung dieser Bedingung entfernt sind, definieren wir eine Funktion $v_\sigma(p_1, p_2) = \max(0, \#_\sigma(p_1, p_2) - 1)$.

\subsection{Symmetrievermeidung}
Für eine Lösung existieren sehr viele äquivalente Lösungen. Innerhalb einer Gruppe ist die Reihenfolge der Golfer beliebig.
Ebenso ist die Reihenfolge der Gruppen innerhalb einer Woche, sowie die Wochen an sich belanglos. Außerdem kann einer neue äquivalente Lösung erzeugt werden, indem man überall die Namen von zwei Golfern tauscht. Wir stellen also fest, dass jede Lösung $(s!)^{g \cdot w} (g!)^w w!(gs)$ symmetrische Lösungen besitzt. 

Um zwei Lösungen unterscheiden zu können, braucht es unterschiedliche Strategien, um diese Symmetrien zu brechen. 
Die Symmetrie innerhalb einer Gruppe wird vermieden, indem man Mengen anstatt Listen verwendet. Diese Mengen werden innerhalb einer Woche anhand ihres kleinsten Elementes geordnet.
\begin{equation}
  \begin{split}
    & \sigma(\che, \omega) \prec \sigma(\che + 1, \omega) \\
    & \Leftrightarrow \min(\sigma(\che, \omega)) < \min(\sigma(\che + 1, \omega))
  \end{split}
\end{equation}
Wochen werden anhand des zweitkleinsten Elementes der ersten Gruppe geordnet.
\begin{equation}
  \begin{split}
    & \omega_i \prec \omega_{i+1} \\
    & \Leftrightarrow \min(\sigma(1, i) \setminus \{ 1 \}) < \min(\sigma(1, i + 1) \setminus \{ 1 \})
  \end{split}
\end{equation}
Die Symmetrien, welche durch Umbenennung enstehen, wurden in dem Artikel nicht weiter behandelt. In \cite{sellmann02} wird dieses Thema weiter behandelt.





\section{Memetischer Lösungsansatz}
  \subsection{Evolution}
Die bereits benannten Symmetrien erschweren den Entwurf eines evolutionären Algorithmus, da schwer entscheidbar ist, ob eine neue Lösung sich überhaupt von den Bisherigen
% 'Bisherigen' wird groß geschrieben, weil es sich in diesem Kontext um eine Nominalisierung handelt
unterscheidet.
Wenn diese Symmetrien bei der Rekombination von zwei Lösungen nicht von vorne herein ausgeschlossen werden können, gewinnt man hierdurch keine Zeit. 
Aus diesen Gründen wird für den hier vorgestellten Algorithmus auch auf die Rekombination verzichtet und wir beschränken uns auf eine Mutation. 

Sei dazu $\sigma$ ein $w$ Wochen langer Spielplan mit $g$ Gruppen zu je $s$ Spielern. Die Darstellung erfolgt über einen String\label{Lösungsstring} $u = t_{11} :: t_{12} :: \cdots :: t_{wg}$ mit $t_{ij} \in \N_{g \cdot s}^s$.

\subsubsection{Beispiel}
Gegeben sei eine Probleminstanz $(2, 2, 3)$. Ein zugehöriger String lautet:
\begin{equation}
\begin{split} 
  u = 1, 2, 3, 4,\quad 4, 1, 2, 3,\quad 1, 3, 2, 4
\end{split}
\end{equation}
Ein Spielplan wird also als Aneinanderreihung von Spielernummern gruppiert nach Woche und Gruppe dargestellt. Diese Darstellung ist äquivalent zu bisher präsentierten mathematischen Modellierung.
Diese Darstellung berücksichtigt zwar keine Symmetrien, allerdings gilt die Einschränkung, dass ein Golfer nur einmal pro Woche eingeteilt werden kann. 
Um eine neue Woche zu erzeugen nutzen wir also eine beliebige Permutation von $\{ 1, \cdots, g \cdot s\}$. Alle Funktionen und Operatoren innerhalb des Algorithmus nutzen intern diese Struktur. 
Der Memetische Algorithmus erzeugt einen Pool an Spielplänen, aus denen Lösungen ausgewählt, mutiert und einem Lernprozess unterzogen werden.

Bei der Mutation werden innerhalb einer Woche $\omega$ zwei Spieler $p_1, p_2$ aus unterschiedlichen Gruppen vertauscht. Die Menge der möglichen Vertauschungen lässt sich beschreiben als:
\begin{equation}
  \mathcal{S}(\sigma) = \{ ((\omega, p_1), (\omega, p_2)) \;|\; \gamma(p_1, \omega) \neq \gamma(p_2, \omega) \}
\end{equation}
Eine ausgewählte Lösung wird eine zufällige Anzahl, deren Mittelwert bei $\omega$ liegt, oft mutiert.

%Jede ausgewählte Lösung wird eine Poisson-Verteilte Anzahl mit dem Parameter $(g \cdot s)^{-1}$ oft wiederholt, sodass im Mittel $w$ mal getauscht wird.
%Die Mutation führt dazu, dass die Suche nicht nur lokale Optima findet, sondern auch

%Es wurden zwei Strategien für den Lernvorgang betrachtet: Baldwin und Lamarck.
%Beim Baldwin'schen Lernen wird eine lokale Verbesserungsfunktion genutzt, deren Fitness beibehalten wird, ihre phänotypischen Änderungen aber verworfen werden. 
%Es wird also eine Lösung daran bewertet, wie gut sie bekommen könnte.
%Im Gegensatz dazu wird bei der Lamarck-Strategie auch die phänotypischen Änderungen weiterverwendet.
%Die Autoren äußern sich leider nicht dazu, wie sie diesen Zusammenhang in ihrem Algorithmus angewandt haben, 
%attestieren der Lamarck-Strategie aber für ihren Algorithmus eine höhere Effizienz.

\subsection{Lokale Suche}
Wie auch bei der Mutation werden bei der lokalen Suche Spieler aus unterschiedlichen Gruppen einer Woche vertauscht. 
Allerdings wird hierbei darauf geachtet, nur diejenigen Spieler zu tauschen, deren Position einen Konflikt zur  Bedingung (\ref{maximale Sozialisation}) bedeutet. 
Wir drücken dies aus mit $v_0((\omega, p)) = \texttt{true}$ genau dann, wenn:
\begin{equation}
  \exists p' \in \sigma(\gamma(p,\omega), \omega): p' \neq p \Rightarrow v_\sigma(p, p') > 1
\end{equation}
Die Menge der hierdurch möglichen Vertauschungen ist:
\begin{equation}
  \mathcal{S}^-(\sigma) = \{ ((\omega, p_1), (\omega, p_2)) \in \mathcal{S} \;|\; v_0((\omega, p_1)) \}
\end{equation}


\subsection{Tabu}
Die lokale Suche wurde durch einen Tabumechanismus ergänzt, welcher sich bereits gemachte Schritte merkt und das wiederholte Tauschen zweier bereits miteinander getauschter Spieler für eine zufällige Anzahl Züge unterbindet.
Wenn allerdings das Tauschen zweier Spieler eine deutliche Verbesserung nach sich zieht, wird die Tabuliste ignoriert.
Dieser Teil besteht aus mehreren Listen für je eine Woche in denen Tupel der Form $(p_1, p_2, i)$ enthalten sind. Wir bezeichnen mit $tabu[\omega]$ die Tabu Liste für die Woche $\omega$. 
Jedes Mal, wenn zwei Spieler getauscht worden sind, wird ein zufälliger Wert $\rho \in [4, 100]$ erzeugt. Dieser gibt an, wie viele Iterationen diese beiden Spieler nicht mehr getauscht werden dürfen. 
Die Summe von $\rho$ und der aktuellen Iteration wird in $i$ gespeichert. Hieraus resultiert eine neue Tauschmenge:
\begin{equation}
\begin{split} 
  \mathcal{S}^t(\sigma, k) = & \{ ((\omega, p_1), (\omega, p_2)) \in \mathcal{S}^- \\
    & \hspace{5 mm} | \; \nexists k' > k: (p_1, p_2, k') \in tabu[\omega] \}
\end{split} 
\end{equation}

Parallel zu dieser Tabu Suche werden allerdings auch Schritte beachtet, die die Lösung besser werden lassen, als die aktuell beste Lösung. Wir nennen die zugehörige Menge $\mathcal{S}^*(\sigma, \sigma*)$:
\begin{equation}
  \mathcal{S}^*(\sigma, \sigma*) = \{ (t_1, t_2) \in \mathcal{S}^- \;|\; f(\sigma [t_1 \leftrightarrow t_2] ) < f(\sigma *) \}
\end{equation}
Hierbei bezeichnet $\sigma^*$ die beste bekannte Lösung und $\sigma [(\omega, p_1) \leftrightarrow (\omega, p_2)]$ bezeichnet die Lösung, bei der in der  Woche $\omega$ Spieler $p_1$ und $p_2$ getauscht wurden.\\
$f(\sigma [t_1 \leftrightarrow t_2])$ ist die Anzahl der Integritätsverletzungen, welche die neue Lösung hat.




\section{Testergebnisse}
  Um die These der Autoren zu prüfen, habe ich eine eigene Implentierung des beschriebenen Algorithmus vorgenommen, wobei ich bemüht war, möglichst genau das beschriebene Verfahren nach zu bilden.
Genau wie die Autoren nutze ich eine Größe von 25 Individuen für den Genpool, aus denen gewählt wird. Die lokale Suche wird mit $maxIter = g \cdot w$ beschränkt, 
damit die Chance besteht, dass die resultierende Lösung sich in jeder Gruppe von der Originalen unterscheidet. 
Ich lasse automatisiert Lösungen für verschiedene Probleminstanzen erzeugen und werde alle Ergebnisse, die bis zu meinem Vortrag erzeugt wurden, veröffentlichen. 
Ich kann verifizieren, dass die Autoren ein Verfahren vorgestellt haben, welches tatsächlich das versprochene Verhalten zeigt.
Über das Zeitverhalten kann ich kaum qualifizierte Aussagen machen, da dies sehr subjektiv und abhängig von der verwendeten Hardware ist. 
Die anspruchsvollste Lösung für eine Instanz, die ich finden konnte, ist die von Alejandro Aguado\cite{agu04} für $(8, 4, 10)$.
Sofern mein Programm eine ebenbürtige Lösung findet, werde ich diese auf Äquivalenz prüfen.



\section{Fazit}
Hier steht, was ich von der ganzen Sache halte und welche Ergebnisse ich aus dieser Arbeit gezogen habe.

\begin{thebibliography}{9}  
  \bibitem{cotta06}
    Carlos Cotta, Ivan Dotu, Antonio J. Fernandez, Pascal Van Hentenryck,
    \emph{Scheduling Social Golfers with Memetic Evolutionary Programming}.
    Springer-Verlag Berlin Heidelberg,
    2006.

  \bibitem{fahle01}
    Fahle, T. Schamberger, S., Sellman, M.
    \emph{Symmetry breaking}.
    In Walsh T., ed.: 7th International Conference on Principles and Practice of Constraint Programming. Volume 2239 of Lecture Notes in Computer Science., Paphos, Cyprus,
    Springer,
    2001 S93-107

  \bibitem{smith01}
    Smith, B.M.
    \emph{Reducing Symmetry in a combinatorial design problem}.
    In: Third International Workshop on the Integration of AI and OR Techniques in Constraint Programming for Combinatorial Optimization Problems,
    2001 S351-359

  \bibitem{sellmann02}
    Sellmann, M., Harvey, W.
    \emph{Heuristic constraint propagation}.,
    In Hetenryck, P. V., ed.: 8th International Conference on Principles and Practice of Constraint Programming. Volume 2470 of Lecture Notes in Computer Science,
    Ithaca, NY, USA
    Springer,
    2002 S738-743

  \bibitem{triska08}
    Triska, Markus and Musliu, Nysret
    \emph{An improved SAT formulation for the social golfer problem},
    In: Annals of Operations Research,
    Springer Verlag,
    2012


    
\end{thebibliography}

\end{document}
