\documentclass[draft]{tcs-seminar}

\usepackage[utf8]{inputenc}  % Erlaubt es, Umlaute etc. zu verwenden (Datei muss UTF-8 Kodierung haben)
\usepackage[OT2, T1]{fontenc}
\usepackage[russian,ngerman]{babel}
\usepackage{url}
\usepackage{amsmath}
\usepackage[autostyle=true,german=quotes]{csquotes}

%\usepackage{breqn}
% convienience
\newcommand{\N}{\mathbb{N}}
\newcommand{\che}{\text{\foreignlanguage{russian}{ч}}}

\begin{document}

\title{Ein Memetischer Algorithmus für das Social Golfer Problem}
\author{Dennis Lindemann}

\begin{abstract}
Das Social Golfer Problem fragt nach einem Spielplan für einen Golfclub, der sich wöchentlich trifft, um in mehreren Gruppen zu golfen, wobei zwei Golfer nie mehr als einmal in der gleichen Gruppe spielen.
Dabei wird die Mitgliederzahl des Clubs, sowie die Gruppengröße und die Anzahl der Wochen variiert.
Die Berechnung einer (neuen) Lösung ist sehr komplex, da zu jeder Lösung eine große Menge an äquivalenten symmetrischen Lösungen existiert.
Das SGP hat sich zu einer Standardanwendung für Benchmarks im Bereich der Constraints Programmierung entwickelt.
Carlos Cotta et al stellen in ihrem Artikel \cite{cotta06} einen neuen Memetischen Algorithmus vor, um dieses Problem zu lösen. 
Sie beschränken sich dabei auf Selektion, Mutation und lokaler (Tabu-)Suche, da eine Rekombination mit hoher Warscheinlichkeit zu symmetrischen Lösungen führen würde.
Die Autoren führen an ihren Beispielalgorithmen verschiedene Performancetests durch und untersuchen den Einfluss einer Baldwinschen und einer Lamarckschen Lernstrategie.

\end{abstract}

\maketitle


\section{Einleitung}
  Das Social Golfer Problem (SGP) findet seine erste Erwähnung in leicht abgewandelter Form schon 1850 im Lady's and Gentleman's Diary, wo Reverend Thomas Kirkman folgende Frage stellte:

\begin{quote}
Fünfzehn junge Mädchen gehen in der Schule sieben Tage lang in Dreiergruppen nebeneinander. Stellen Sie die Gruppen so zusammen, dass zwei Mädchen nicht zweimal zusammen gehen.
\end{quote}
Diese Frage ist etwas spezieller als diejenige, welche in diesen Artikel untersucht wird:
\begin{quote}
$n$ Golfer spielen in $g$ Gruppen der Größe $s$ $w$ Wochen lang. Erstellen Sie einen Spielplan, in dem zwei Spieler maximal einmal zusammen spielen.
\end{quote}
Die allgemeine Variante hat internationale Bekanntheit erlangt, nachdem sie im Mai 1998 in der Newsgroup \texttt{sci.op-research} veröffentlicht wurde.

Das allgemeine SGP ist für einige Probleminstanzen hinreichend komplex, dass in absehbarer Zeit keine Lösung erwartet wird. Es liegt die Vermutung nahe, dass es sich hierbei um ein NP-vollständiges Problem handelt.
Erschwerend kommt hinzu, dass Lösungen für das SGP eine hohe Symmetriedichte aufweisen. 
Beispielsweise können innerhalb eines Spielplanes die Wochen getauscht werden, sowie die Gruppen innerhalb einer Woche. 
Auch können zwei Spieler vertauscht werden, ohne dass sich die Lösung prinzipiell von der Alten unterscheidet. 
Jeder Lösungsansatz, der sich nicht eingehend mit dieser Erschwernis befasst, ist dazu verdammt, immer wieder redundante Lösungen zu finden.

Vor diesem Hintergrund ist ein Algorithmus, der für eine Probleminstanz $(g,s,w)$ in vertretbarer Zeit einen Spielplan erzeugt, von großem Interesse.
Mehrere Ansätze, zum Beispiel aus der Constraints Programmierung (z.B. \cite{fahle01}, \cite{smith01}, \cite{sellmann02}) oder mit Hilfe von SAT Solvern (z.B. \cite{triska08}), 
wurden schon verfolgt. Allerdings war unter den bisherigen Algorithmen noch keiner, mit einem memetischen Ansatz.
Die Kombination von Selektion, Mutation und lokaler Suche, welche als \enquote{memetischer Algorithmus} bezeichnet wird, führt zu einer Performance, die sich mit den Alternativen messen kann.





\section{Mathematische Modellierung}
    
Hier wird stehen, in welcher Art die Mathematischen Objekte definiert sind.

\subsection{Symmetrievermeidung}
Lösungen für das SGP sind hochsymmetrisch, dass heißt: Zwei Wochenpläne können komplett anders ausehen und trotzdem die gleiche Lösung repräsentieren. In diesem Absatz gehe ich darauf ein, wie Symmetrien vermieden werden.

\section{Memetischer Algorithmus}
  \subsection{Evolution}
    Die Mutationen kann man als einen weiten Sprung zu einer anderen Lösung ansehen. (Makrooptimierung)
  \subsection{Lokale Suche}
    Die Tabu Suche ist eher ein kleiner Sprung, um eine bestehende (ungültige) Lösung zu verfeinern. (Mikrooptimierung)

\section{Testergebnisse}
Hier werden die Ergebnisse meiner Performancetests vorgestellt und mit den Ergebnissen aus dem Paper verglichen. Ein besonderes Augenmerk werde ich hierbei auf Laufzeit und Ressourcenverbrauch im Sinne der Landau Notation legen.



\section{Fazit}
Hier steht, was ich von der ganzen Sache halte und welche Ergebnisse ich aus dieser Arbeit gezogen habe.

\begin{thebibliography}{9}  
  \bibitem{cotta06}
    Carlos Cotta, Ivan Dotu, Antonio J. Fernandez, Pascal Van Hentenryck,
    \emph{Scheduling Social Golfers with Memetic Evolutionary Programming}.
    Springer-Verlag Berlin Heidelberg,
    2006.

  \bibitem{fahle01}
    Fahle, T. Schamberger, S., Sellman, M.
    \emph{Symmetry breaking}.
    In Walsh T., ed.: 7th International Conference on Principles and Practice of Constraint Programming. Volume 2239 of Lecture Notes in Computer Science., Paphos, Cyprus,
    Springer,
    2001 S93-107

  \bibitem{smith01}
    Smith, B.M.
    \emph{Reducing Symmetry in a combinatorial design problem}.
    In: Third International Workshop on the Integration of AI and OR Techniques in Constraint Programming for Combinatorial Optimization Problems,
    2001 S351-359

  \bibitem{sellmann02}
    Sellmann, M., Harvey, W.
    \emph{Heuristic constraint propagation}.,
    In Hetenryck, P. V., ed.: 8th International Conference on Principles and Practice of Constraint Programming. Volume 2470 of Lecture Notes in Computer Science,
    Ithaca, NY, USA
    Springer,
    2002 S738-743

  \bibitem{triska08}
    Triska, Markus and Musliu, Nysret
    \emph{An improved SAT formulation for the social golfer problem},
    In: Annals of Operations Research,
    Springer Verlag,
    2012


    
\end{thebibliography}

\end{document}
